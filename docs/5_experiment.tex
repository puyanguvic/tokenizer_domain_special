\section{Experiments}
\label{sec:experiments}

We evaluate the Domain-Specific Tokenization (DST) framework across multiple structured and semi-structured domains, focusing on the design, setup, and evaluation methodology.
The experiments are designed to assess three main aspects:

\begin{enumerate}
  \item \textbf{Efficiency:} sequence-length compression and throughput improvement relative to subword baselines.
  \item \textbf{Reliability:} guaranteed invertibility and deterministic decoding across all domains.
  \item \textbf{Compatibility:} stability of downstream model performance on general-language benchmarks.
\end{enumerate}

\subsection{Experimental Setup}

All tokenizers are trained under comparable vocabulary budgets (\placeholder{e.g., 16K–32K}).
Evaluation spans four representative structured data domains:

\begin{itemize}
  \item \textbf{Source Code:} \placeholder{Python and Java snippets from open repositories.}
  \item \textbf{Configuration Logs:} \placeholder{Kubernetes YAML, Nginx, and other service configurations.}
  \item \textbf{Network Protocols:} \placeholder{HTTP requests and response traces.}
  \item \textbf{Biosequences:} \placeholder{DNA and protein sequences (k-mer representations).}
\end{itemize}

For fair comparison, each tokenizer variant—BPE, Unigram, Byte-level, and DST—is applied to the same pretrained Transformer backbone (\placeholder{e.g., BERT-base or RoBERTa-base}) and trained under identical hyperparameters.
General-language tasks (\placeholder{e.g., MMLU, HellaSwag}) are included to test cross-domain generalization.
All experiments are conducted on standardized hardware (\placeholder{e.g., A100 40GB}) under identical training and inference configurations.

\subsection{Evaluation Metrics}

We report the following metrics to quantify performance:

\begin{itemize}
  \item \textbf{Tokens/Sequence:} average number of tokens per input sequence (\placeholder{lower is better}).
  \item \textbf{Entropy:} mean per-token cross-entropy on held-out data normalized to a baseline tokenizer.
  \item \textbf{Throughput:} number of processed tokens per second during inference (\placeholder{higher is better}).
  \item \textbf{Memory Usage:} peak GPU VRAM consumption during inference (\placeholder{lower is better}).
  \item \textbf{General-Language Performance $\Delta$:} accuracy or perplexity difference on open-domain benchmarks relative to baseline models.
\end{itemize}

\begin{table*}[!t]
  \centering
  \small
  \begin{tabular}{lcccccc}
    \toprule
    Tokenizer & Invertibility & Tokens/Seq & Entropy & Throughput & Memory & Gen.~Perf.~$\Delta$ \\
    \midrule
    \placeholder{BPE Baseline} & \placeholder{✓ / ✗} & \placeholder{--} & \placeholder{--} & \placeholder{--} & \placeholder{--} & \placeholder{--} \\
    \placeholder{Unigram (SentencePiece)} & \placeholder{✓ / ✗} & \placeholder{--} & \placeholder{--} & \placeholder{--} & \placeholder{--} & \placeholder{--} \\
    \placeholder{Byte-Level (ByT5)} & \placeholder{✓ / ✗} & \placeholder{--} & \placeholder{--} & \placeholder{--} & \placeholder{--} & \placeholder{--} \\
    \placeholder{Structured Tokenization (Xu et al., 2024)} & \placeholder{✓ / ✗} & \placeholder{--} & \placeholder{--} & \placeholder{--} & \placeholder{--} & \placeholder{--} \\
    \textbf{DST (Ours)} & \textbf{✓ (Exact)} & \placeholder{--} & \placeholder{--} & \placeholder{--} & \placeholder{--} & \placeholder{--} \\
    \bottomrule
  \end{tabular}
  \caption{Cross-domain evaluation placeholders for tokenization methods. Metrics normalized to baseline values.}
  \label{tab:main}
\end{table*}

\subsection{Qualitative Evaluation}

We further examine qualitative aspects of DST through case studies and visualization:

\begin{itemize}
  \item \textbf{Syntactic Boundary Alignment:} visualization of token boundaries on structured text (e.g., YAML, HTML, HTTP logs).
  \item \textbf{Invertibility Verification:} round-trip reconstruction test confirming $\kappa(\tau(x)) = x$ for all samples in held-out corpora.
  \item \textbf{Deployment Integration:} runtime analysis in real-world pipelines such as \placeholder{configuration classification or protocol inspection}.
\end{itemize}

\begin{figure}[t]
  \centering
  \setlength{\fboxsep}{0pt}\fbox{\rule{0pt}{100pt}\rule{220pt}{0pt}}
  \caption{Placeholder for sequence length vs. latency across domains.}
  \label{fig:dst_efficiency}
\end{figure}

\subsection{Ablation and Sensitivity Design}

Ablation experiments are planned to isolate the effects of different design components:

\begin{itemize}
  \item Removing grammar priors to evaluate the role of syntactic constraints.
  \item Varying vocabulary size ($|\mathcal{V}| \in \{8\mathrm{K}, 16\mathrm{K}, 32\mathrm{K}\}$) to study compression trade-offs.
  \item Adjusting scoring coefficients $(\alpha, \beta, \gamma, \eta)$ in the hybrid candidate-selection objective.
  \item Disabling DFST determinism to measure its contribution to runtime and reliability.
\end{itemize}

\begin{table}[t]
  \centering
  \small
  \begin{tabular}{lccc}
    \toprule
    Setting & Tokens/Seq & Boundary Violations & Invertibility \\
    \midrule
    \placeholder{Full DST (Default)} & \placeholder{--} & \placeholder{--} & \placeholder{✓} \\
    \placeholder{No Grammar Priors} & \placeholder{--} & \placeholder{--} & \placeholder{✓ / ✗} \\
    \placeholder{Random Merge Baseline} & \placeholder{--} & \placeholder{--} & \placeholder{✗} \\
    \placeholder{Byte-Level Tokenization} & \placeholder{--} & \placeholder{--} & \placeholder{✓} \\
    \bottomrule
  \end{tabular}
  \caption{Ablation placeholders for different configurations.}
  \label{tab:ablation}
\end{table}

\subsection{Planned Analysis and Reporting}

The final analysis will include:

\begin{itemize}
  \item Correlation between token-count reduction and end-to-end latency.
  \item Visualization of token-boundary alignment under different domain grammars.
  \item Comparative memory/throughput scaling across batch sizes.
  \item Qualitative reconstruction examples confirming exact reversibility.
\end{itemize}

All numerical results are left as placeholders to be filled once empirical evaluation is complete.
This section establishes the experimental design, evaluation protocol, and reporting structure for reproducible measurement of DST’s efficiency, reliability, and compatibility.
